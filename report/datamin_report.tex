\documentclass{article}
\usepackage{graphicx}
\usepackage[english,greek]{babel}
\usepackage[utf8x]{inputenc}
\usepackage{amsmath}
\usepackage{relsize}
\usepackage{enumerate}
\usepackage[parfill]{parskip}
\usepackage{graphicx}
\usepackage{listings}

\makeatletter
\renewcommand*\env@matrix[1][*\c@MaxMatrixCols c]{%
  \hskip -\arraycolsep
  \let\@ifnextchar\new@ifnextchar
  \array{#1}}
\makeatother

\begin{document}

\title{\vspace{-3.5cm}\textbf{Τεχνικές Εξόρυξης Δεδομένων \\ \textlatin{Project \#}1}}
\author{Λάμπρου Ιωάννης \\1115201400088\\\\ Στεφανίδης - Βοζίκης Κωνσταντίνος \\1115201400192}

\maketitle
\section{\textlatin{WordCloud}}
Για την δημιουργία των \textlatin{wordclouds} κάθε κατηγορίας χρησιμοποιήθηκε το κείμενο από όλα τα άρθρα
που υπήρχαν στο \textlatin{train\_set.csv}. Τα \textlatin{wordclouds} που δημιουργήθηκαν ανά κατηγορία είναι: \\
\subsection*{\textlatin{Politics}}
\includegraphics[scale=0.6]{Poli}
\subsection*{\textlatin{Film}}
\includegraphics[scale=0.6]{Film}
\subsection*{\textlatin{Football}}
\includegraphics[scale=0.6]{Foot}
\subsection*{\textlatin{Business}}
\includegraphics[scale=0.6]{Busi}
\subsection*{\textlatin{Technology}}
\includegraphics[scale=0.6]{Techno}

Αξίζει να σημειωθεί πως για να δημιουργηθούν αυτά τα \textlatin{wordclouds}, θεωρήθηκαν ως \textlatin{stopwords} 
τα \textlatin{english stopwords} της βιβλιοθήκης \textlatin{sklearn}
στις οποίες προστέθηκαν οι λέξεις \textlatin{'say', 'said', 'th', 'it,} και \textlatin{'ha'}, τις οποίες κρίναμε ως επιπλέον 
\textlatin{stopwords}, μετά από εμφάνισή τους στα \textlatin{wordclouds} αυτά. 

\section{Υλοποίηση Κατηγοριοποίησης \textlatin{(Classification)}}
Για την κατηγοριοποίηση δοκιμάστηκαν όλες οι ζητούμενες μέθοδοι κατηγοριοποίησης, δηλαδή οι 
\textlatin{Support Vector Machines (SVM), Random Forests, Multinomial Naive Bayes}, καθώς και 
έγινε δική μας υλοποίηση της μεθόδου \textlatin{K-Nearest Neighbor}.\\\\
Ακόμα, για την αξιολόγηση της απόδοσης όλων αυτών των μεθόδων χρησιμοποιήθηκε
\textlatin{10-fold Cross Validation} για όλες τις ζητούμενες μετρικές της εκφώνησης
\textlatin{(Precision, Recall, F-Measure, Accuracy)} που παρέχει η βιβλιοθήκη \textlatin{scikit}.\\
Όπως ζητήθηκε, για την προεπεξεργασία των δεδομένων χρησιμοποιήθηκε η τεχνική 
\textlatin{Latent Semantic Indexing (LSI)}. Κρατώντας τους παραπάνω \textlatin{classifiers} 
(\textlatin{SVM, Random Forests, Multinomial Naive Bayes, KNN}) σταθερούς, με το 
\textlatin{component number} να παίρνει τιμές 50,100,200, και με \textlatin{10-fold Cross Validation}
παίρνουμε τις τιμές για \textlatin{Accuracy}:\\




Ακόμα, για το \textlatin{Classification}, η πληροφορία του τίτλου και η πληροφορία που παίρνουμε από το κείμενο
έχουν ίδιο βάρος, 50-50.\\\\

Επίσης, όπως προαναφέρθηκε, χρησιμοποιήθηκε δική μας υλοποίηση για την μέθοδο \textlatin{K-Nearest Neighbors},
\textlatin{(nearest\_neighbor\_validation()} η οποία καλεί την\\ \textlatin{find\_k\_nearest()} στον κώδικά μας)
ενώ και η τελική επιλογή για \textlatin{predicted label} έγινε με \textlatin{majority voting}.\\

Τέλος, στον κώδικά μας τρέχουμε την μέθοδο \textlatin{Support Vector Machines} χωρίς παραμέτρους, αλλά
και με τις βέλτιστες παραμέτρους \textlatin{kernel='rbf', C=10, gamma=1} τις οποίες και βρίκαμε με τη χρήση
της \textlatin{GridSearchCV()}. (μέσω της \textlatin{find\_parameters()} στον κώδικά μας) 






\end{document}
